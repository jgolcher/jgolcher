\PassOptionsToPackage{unicode=true}{hyperref} % options for packages loaded elsewhere
\PassOptionsToPackage{hyphens}{url}
%
\documentclass[]{article}
\usepackage{lmodern}
\usepackage{amssymb,amsmath}
\usepackage{ifxetex,ifluatex}
\usepackage{fixltx2e} % provides \textsubscript
\ifnum 0\ifxetex 1\fi\ifluatex 1\fi=0 % if pdftex
  \usepackage[T1]{fontenc}
  \usepackage[utf8]{inputenc}
  \usepackage{textcomp} % provides euro and other symbols
\else % if luatex or xelatex
  \usepackage{unicode-math}
  \defaultfontfeatures{Ligatures=TeX,Scale=MatchLowercase}
\fi
% use upquote if available, for straight quotes in verbatim environments
\IfFileExists{upquote.sty}{\usepackage{upquote}}{}
% use microtype if available
\IfFileExists{microtype.sty}{%
\usepackage[]{microtype}
\UseMicrotypeSet[protrusion]{basicmath} % disable protrusion for tt fonts
}{}
\IfFileExists{parskip.sty}{%
\usepackage{parskip}
}{% else
\setlength{\parindent}{0pt}
\setlength{\parskip}{6pt plus 2pt minus 1pt}
}
\usepackage{hyperref}
\hypersetup{
            pdftitle={Intro to R in 15 Steps},
            pdfborder={0 0 0},
            breaklinks=true}
\urlstyle{same}  % don't use monospace font for urls
\usepackage[margin=1in]{geometry}
\usepackage{color}
\usepackage{fancyvrb}
\newcommand{\VerbBar}{|}
\newcommand{\VERB}{\Verb[commandchars=\\\{\}]}
\DefineVerbatimEnvironment{Highlighting}{Verbatim}{commandchars=\\\{\}}
% Add ',fontsize=\small' for more characters per line
\usepackage{framed}
\definecolor{shadecolor}{RGB}{248,248,248}
\newenvironment{Shaded}{\begin{snugshade}}{\end{snugshade}}
\newcommand{\AlertTok}[1]{\textcolor[rgb]{0.94,0.16,0.16}{#1}}
\newcommand{\AnnotationTok}[1]{\textcolor[rgb]{0.56,0.35,0.01}{\textbf{\textit{#1}}}}
\newcommand{\AttributeTok}[1]{\textcolor[rgb]{0.77,0.63,0.00}{#1}}
\newcommand{\BaseNTok}[1]{\textcolor[rgb]{0.00,0.00,0.81}{#1}}
\newcommand{\BuiltInTok}[1]{#1}
\newcommand{\CharTok}[1]{\textcolor[rgb]{0.31,0.60,0.02}{#1}}
\newcommand{\CommentTok}[1]{\textcolor[rgb]{0.56,0.35,0.01}{\textit{#1}}}
\newcommand{\CommentVarTok}[1]{\textcolor[rgb]{0.56,0.35,0.01}{\textbf{\textit{#1}}}}
\newcommand{\ConstantTok}[1]{\textcolor[rgb]{0.00,0.00,0.00}{#1}}
\newcommand{\ControlFlowTok}[1]{\textcolor[rgb]{0.13,0.29,0.53}{\textbf{#1}}}
\newcommand{\DataTypeTok}[1]{\textcolor[rgb]{0.13,0.29,0.53}{#1}}
\newcommand{\DecValTok}[1]{\textcolor[rgb]{0.00,0.00,0.81}{#1}}
\newcommand{\DocumentationTok}[1]{\textcolor[rgb]{0.56,0.35,0.01}{\textbf{\textit{#1}}}}
\newcommand{\ErrorTok}[1]{\textcolor[rgb]{0.64,0.00,0.00}{\textbf{#1}}}
\newcommand{\ExtensionTok}[1]{#1}
\newcommand{\FloatTok}[1]{\textcolor[rgb]{0.00,0.00,0.81}{#1}}
\newcommand{\FunctionTok}[1]{\textcolor[rgb]{0.00,0.00,0.00}{#1}}
\newcommand{\ImportTok}[1]{#1}
\newcommand{\InformationTok}[1]{\textcolor[rgb]{0.56,0.35,0.01}{\textbf{\textit{#1}}}}
\newcommand{\KeywordTok}[1]{\textcolor[rgb]{0.13,0.29,0.53}{\textbf{#1}}}
\newcommand{\NormalTok}[1]{#1}
\newcommand{\OperatorTok}[1]{\textcolor[rgb]{0.81,0.36,0.00}{\textbf{#1}}}
\newcommand{\OtherTok}[1]{\textcolor[rgb]{0.56,0.35,0.01}{#1}}
\newcommand{\PreprocessorTok}[1]{\textcolor[rgb]{0.56,0.35,0.01}{\textit{#1}}}
\newcommand{\RegionMarkerTok}[1]{#1}
\newcommand{\SpecialCharTok}[1]{\textcolor[rgb]{0.00,0.00,0.00}{#1}}
\newcommand{\SpecialStringTok}[1]{\textcolor[rgb]{0.31,0.60,0.02}{#1}}
\newcommand{\StringTok}[1]{\textcolor[rgb]{0.31,0.60,0.02}{#1}}
\newcommand{\VariableTok}[1]{\textcolor[rgb]{0.00,0.00,0.00}{#1}}
\newcommand{\VerbatimStringTok}[1]{\textcolor[rgb]{0.31,0.60,0.02}{#1}}
\newcommand{\WarningTok}[1]{\textcolor[rgb]{0.56,0.35,0.01}{\textbf{\textit{#1}}}}
\usepackage{graphicx,grffile}
\makeatletter
\def\maxwidth{\ifdim\Gin@nat@width>\linewidth\linewidth\else\Gin@nat@width\fi}
\def\maxheight{\ifdim\Gin@nat@height>\textheight\textheight\else\Gin@nat@height\fi}
\makeatother
% Scale images if necessary, so that they will not overflow the page
% margins by default, and it is still possible to overwrite the defaults
% using explicit options in \includegraphics[width, height, ...]{}
\setkeys{Gin}{width=\maxwidth,height=\maxheight,keepaspectratio}
\setlength{\emergencystretch}{3em}  % prevent overfull lines
\providecommand{\tightlist}{%
  \setlength{\itemsep}{0pt}\setlength{\parskip}{0pt}}
\setcounter{secnumdepth}{0}
% Redefines (sub)paragraphs to behave more like sections
\ifx\paragraph\undefined\else
\let\oldparagraph\paragraph
\renewcommand{\paragraph}[1]{\oldparagraph{#1}\mbox{}}
\fi
\ifx\subparagraph\undefined\else
\let\oldsubparagraph\subparagraph
\renewcommand{\subparagraph}[1]{\oldsubparagraph{#1}\mbox{}}
\fi

% set default figure placement to htbp
\makeatletter
\def\fps@figure{htbp}
\makeatother


\title{Intro to R in 15 Steps}
\author{}
\date{\vspace{-2.5em}}

\begin{document}
\maketitle

The goal of this tutorial is to introduce you to one of the most
commonly used programming languages in computational biology:
\textbf{R}. Throughout this tutorial, you can think of learning
\textbf{R} as learning how to use a specialized smartphone, and learning
the specific \textbf{R packages} as the various apps designed to help
you do different tasks computational biologists do.

\hypertarget{step-1-create-an-account-with-rstudio-cloud}{%
\subsubsection{Step 1: Create an account with RStudio
Cloud}\label{step-1-create-an-account-with-rstudio-cloud}}

\textbf{R \& R Studio are free!} For this tutorial, I recommend you use
R Studio Cloud. To create a free account just access this website
\emph{\url{https://login.rstudio.cloud}} and create a new account using
your preferred email address.

\hypertarget{step-1b-download-and-install-r-r-studio-alternative}{%
\paragraph{Step 1b: Download and install R + R studio
alternative}\label{step-1b-download-and-install-r-r-studio-alternative}}

\textbf{R is free and will always be!} If after this workshop, you would
like to go one step further, you can download R by accessing this
website: \emph{\url{http://www.r-project.org/}} The download link will
take you to a list of CRAN mirrors -- pick a mirror: e.g.~South Africa
(for a list of the mirrors
\emph{\url{https://cran.r-project.org/mirrors.html}}).

Download R studio - free option;
\emph{\url{http://www.rstudio.com/ide/download/}}.

R studio is composed by \textbf{4 panels}: \textbf{1=}Top Left,
\textbf{2=}Top Right, \textbf{3=}Bottom Left, and \textbf{4=}Bottom
right. Note taking happens in \textbf{Panel 1}. \textbf{Panel 2} has
``your environment'', every time you open R, you read data or create any
kind of object in R it will be listed in your
\textbf{\emph{Environment}} tab. Any code you run in \textbf{Panel 3}
will appear also in Panel 2, under \textbf{\emph{History}} and anything
that you visualize and more will be in \textbf{Panel 4}. We will delve
more into the details soon.

\emph{You will then need to set your working directory in your local
computer if you chose the option to download R.}

A working directory is the folder in your computer where all available
data are \& results will be saved.To find out the current working
directory, type getwd() and hit ENTER.To set a new working directory,
you will specify the computer path to a folder, for example
\emph{setwd(``\textasciitilde{}/Dropbox/USB'')}. You can also use the
non-programming option by following the drop-down menu on R Studio:
click on the tab called \emph{Session} -\textgreater{} \emph{Set Working
Directory} -\textgreater{} \emph{Choose Directory} and just select the
folder of your choice in your computer.

\hypertarget{step-2-learn-the-help-command}{%
\subsubsection{Step 2: Learn the Help
Command}\label{step-2-learn-the-help-command}}

\textbf{\emph{Practice makes perfect}} A way to learn is by deciphering
error messages! So, if you get an error message, which is the output
showing on red font, don't panic! Computational biologists are
constantly figuring out what error messages mean and why do we get them.
An effective way to do this is to \textbf{google} your error message.
Often someone has already found a solution and there is a ton of help
online! Now since we are learning how to code, we can ask R for help
using a specific function in a package by calling in R's built-in help
function, or by directly typing a question mark before the function you
are trying to use.

\begin{Shaded}
\begin{Highlighting}[]
\CommentTok{#You want to learn what the function "help()" does in R}
\NormalTok{?}\KeywordTok{help}\NormalTok{()}
\KeywordTok{help}\NormalTok{() }
\end{Highlighting}
\end{Shaded}

\textbf{\emph{Question 1:}} How would you access built-in help about the
R package \emph{dplyr}, this is a commonly used package in data
wrangling, management or rearranging
(\emph{\url{https://dplyr.tidyverse.org/}}).\\
\textbf{\emph{A/}}

\hypertarget{step-3-do-basic-calculations}{%
\subsubsection{Step 3: Do basic
calculations}\label{step-3-do-basic-calculations}}

R is like a smartphone, and its built-in package called \textbf{base}
can do everything calculators do \& much more! Let's explore using R as
a calculator.

Let's use R to do some simple calculations:

\begin{Shaded}
\begin{Highlighting}[]
\CommentTok{#Sum}
\DecValTok{3} \OperatorTok{+}\StringTok{ }\DecValTok{3}

\CommentTok{#Exponential}
\DecValTok{2}\OperatorTok{^-}\DecValTok{7}

\CommentTok{#log function}
\NormalTok{?}\KeywordTok{log}\NormalTok{() }\CommentTok{#add a question mark before a function to learn about what it does.}
\KeywordTok{log}\NormalTok{(}\DecValTok{2}\NormalTok{)}
\end{Highlighting}
\end{Shaded}

\textbf{\emph{Question 2:}} What type of logarithm (e.g.~natural log,
common log-base 10, or binary logarithm) does the log function
computes?\\
\textbf{\emph{A/}}

\textbf{\emph{But then why using R and not a calculator?}} In R, anyone
can create so called \textbf{functions} to create plots, maps, figures,
organize data bases, conduct statistical analyses, make presentations,
build websites, analyze whole genome sequences, among others.

\hypertarget{step-4-familiarize-yourself-with-packages-and-versions}{%
\subsubsection{Step 4: Familiarize yourself with packages and
versions}\label{step-4-familiarize-yourself-with-packages-and-versions}}

Functions are compiled into \textbf{packages} (remember to think of
these as the apps in a smartphone) published into the Comprehensive R
Archive Network (CRAN), R's central software repository hosted by the R
foundation. In March 2022, there were 18,948 R packages made available
through CRAN. (\emph{\url{https://cran.r-project.org/web/packages/}})

When using R or R studio, you can find all the built-in packages under
\textbf{Packages} in Panel 4. To search for a specific package you can
select ``Install'', this opens a new window in which you may select the
CRAN repository (recommended option) and search the name of the package
by typing the name of it under \textbf{Packages (separate multiple with
space or comma)}. Make sure that under ``Install to Library''
-\textgreater{} you use the default location to install R packages and
double check that the box for ``install dependencies'' is ticked. We can
also install packages by coding in the console (\textbf{Panel 3}).

For instance search the package called \textbf{ggplot2} used to create
elegant graphs/plots using your data. The amount of freely available
tutorials varies a lot among packages, some plotting packages like
ggplot have lots of tutorials online whereas more specific packages will
have sparse info. Yet, almost all packages have a user guide available
online or by simply clicking on the name of the package in Panel 4,
under the \textbf{Packages} tab, or the \textbf{help()} function we
examined earlier.

\begin{Shaded}
\begin{Highlighting}[]
\CommentTok{#Example installing packages in R}
\KeywordTok{install.packages}\NormalTok{(}\StringTok{"ggplot2"}\NormalTok{) }\CommentTok{#used to create nice plots}
\KeywordTok{install.packages}\NormalTok{(}\StringTok{"dplyr"}\NormalTok{) }\CommentTok{#used to do data management}
\KeywordTok{install.packages}\NormalTok{(}\StringTok{"tidyr"}\NormalTok{) }\CommentTok{#used to tidy messy data}
\KeywordTok{install.packages}\NormalTok{(}\StringTok{"tibble"}\NormalTok{) }\CommentTok{#used together with dplyr to make simple frames}
\KeywordTok{install.packages}\NormalTok{(}\StringTok{"explore"}\NormalTok{) }\CommentTok{#used to do exploratory analyses}
\KeywordTok{install.packages}\NormalTok{(}\StringTok{"patchwork"}\NormalTok{) }\CommentTok{#creates multiplots / composer of plots}
\KeywordTok{library}\NormalTok{(}\StringTok{"ggplot2"}\NormalTok{) }\CommentTok{#Running the library() command "activates" the packages}
\KeywordTok{library}\NormalTok{(}\StringTok{"dplyr"}\NormalTok{)}
\KeywordTok{library}\NormalTok{(}\StringTok{"tidyr"}\NormalTok{)}
\KeywordTok{library}\NormalTok{(}\StringTok{"tibble"}\NormalTok{)}
\KeywordTok{library}\NormalTok{(}\StringTok{"explore"}\NormalTok{)}
\KeywordTok{library}\NormalTok{(}\StringTok{"patchwork"}\NormalTok{)}
\end{Highlighting}
\end{Shaded}

It will become tricky to manage packages if these require different R
versions (just like some apps in your smartphones that stopped being
supported in old smartphones). It is important to keep the packages you
often need organized and only load those that you will use, to avoid any
problems. Also, since the community is actively developing \textbf{free}
packages, there ways to install these are getting easier and easier.

\hypertarget{step-5-familiarize-yourself-with-objects-in-r}{%
\subsubsection{Step 5: Familiarize yourself with objects in
R}\label{step-5-familiarize-yourself-with-objects-in-r}}

R programming language recognizes a handful of fundamental \emph{units}
used to perform different actions, we call those \emph{objects}. The
main types of objects in r are:

\textbf{Functions:} In R, a function is an object that we will
\emph{code}. Think of this as a task you want the computer to do for
you. We will use functions by specifying instructions using
\emph{arguments} (inside of the parentheses) that may be necessary for
the function to accomplish the actions. The basic syntax of a function
can look as simple as this: \emph{help()} or \emph{log(x,base=y)}.

\textbf{Vectors:} Vectors are variables, or series of values of the same
class, they can be continuous (``numeric''), categorical (``factor'')",
etc.

\textbf{Data Frames and Tibbles:} A data frame is a matrix composed by
variables, but formatting variables as a data frame allows handling
versatility in your R studio environment. Many functions require
formatting your data as data frame. We often need to try out different
ways to re-organize our data, so to do this we will take advantage of a
new, simpler and versatile version of the \emph{data frame} object
called the \emph{tibble} to be used along with the package \emph{dplyr}.

\textbf{List:} A list can comprise all sorts of object types. Few
functions require data as lists.

\textbf{Other:} Other less frequently used objects include: scalars (a
single value), matrices (similar to data frame, but allow objects with
varying lengths).

Let's explore these in R:

\begin{Shaded}
\begin{Highlighting}[]
\CommentTok{#Object}
\NormalTok{x <-}\StringTok{ }\DecValTok{3} \OperatorTok{+}\StringTok{ }\DecValTok{5}
\NormalTok{x}

\CommentTok{#Vector}
\NormalTok{y <-}\StringTok{ }\KeywordTok{c}\NormalTok{(}\DecValTok{1}\NormalTok{, }\DecValTok{2}\NormalTok{, }\DecValTok{3}\NormalTok{, }\DecValTok{4}\NormalTok{, }\DecValTok{5}\NormalTok{) }
\NormalTok{y}

\KeywordTok{seq}\NormalTok{(}\DecValTok{1}\NormalTok{,}\DecValTok{10}\NormalTok{, }\DataTypeTok{by =} \DecValTok{1}\NormalTok{)}\CommentTok{#Creates a sequence following specific rules }

\NormalTok{vector1<-}\KeywordTok{seq}\NormalTok{(}\DecValTok{1}\NormalTok{, }\DecValTok{10}\NormalTok{,  }\DataTypeTok{by =} \DecValTok{1}\NormalTok{)}
\NormalTok{vector2<-}\KeywordTok{seq}\NormalTok{(}\DecValTok{1}\NormalTok{, }\DecValTok{20}\NormalTok{, }\DataTypeTok{by =} \DecValTok{2}\NormalTok{)}
\NormalTok{vector3<-}\KeywordTok{seq}\NormalTok{(}\DecValTok{1}\NormalTok{, }\DecValTok{30}\NormalTok{,  }\DataTypeTok{by =} \DecValTok{3}\NormalTok{)}
\NormalTok{vector4<-}\KeywordTok{c}\NormalTok{(}\StringTok{"A"}\NormalTok{, }\StringTok{"B"}\NormalTok{, }\StringTok{"C"}\NormalTok{, }\StringTok{"D"}\NormalTok{, }\StringTok{"E"}\NormalTok{, }\StringTok{"F"}\NormalTok{,}\StringTok{"G"}\NormalTok{, }\StringTok{"H"}\NormalTok{, }\StringTok{"I"}\NormalTok{, }\StringTok{"J"}\NormalTok{)}

\CommentTok{#Data frame}
\NormalTok{df1<-}\KeywordTok{data.frame}\NormalTok{(vector1,vector2,vector3, vector4)}

\CommentTok{#Tibble}
\NormalTok{tibble1<-tibble}\OperatorTok{::}\KeywordTok{tibble}\NormalTok{(df1)}

\CommentTok{#List}
\NormalTok{list1<-}\StringTok{ }\KeywordTok{list}\NormalTok{(df1, vector1, vector2, vector3, vector4, x, y) }

\CommentTok{#Note that giving a name (e.g, "df1") will have downstream implications.}
\CommentTok{#It is important to give your objects unique, intuitive, short & simple names.}
\end{Highlighting}
\end{Shaded}

\textbf{\emph{Question 3:}} What is an advantage and a disadvantage of
using a list?\\
\textbf{\emph{A/}}

\hypertarget{step-6-familiarize-yourself-with-functions-in-r}{%
\subsubsection{Step 6: Familiarize yourself with functions in
R}\label{step-6-familiarize-yourself-with-functions-in-r}}

Here try to find a few more of the many built-in functions in r, for
example:

\begin{Shaded}
\begin{Highlighting}[]
\CommentTok{#Calculate mean}
\NormalTok{?mean}
\KeywordTok{mean}\NormalTok{(vector1)}
\CommentTok{#Sum values in a vector using}
\NormalTok{?sum}
\KeywordTok{sum}\NormalTok{(vector2)}

\CommentTok{#If we want to build a data frame we *need* to have a set of vectors all of the same length }
\CommentTok{#Figure out length of a vector:}
\KeywordTok{length}\NormalTok{(vector3)}
\CommentTok{#Retrieve unique values in vector}
\KeywordTok{unique}\NormalTok{(vector4)}
\CommentTok{#Sort values in a vector }
\NormalTok{df1}\OperatorTok{$}\NormalTok{vector5<-}\KeywordTok{sort}\NormalTok{(vector4, }\DataTypeTok{decreasing =} \OtherTok{TRUE}\NormalTok{)}
\CommentTok{#Sample 6 values at random from a numeric vector}
\KeywordTok{sample}\NormalTok{(vector3, }\DecValTok{3}\NormalTok{)}
\CommentTok{#Sample 4 letters at random from a character vector}
\KeywordTok{sample}\NormalTok{(vector4, }\DecValTok{4}\NormalTok{)}
\end{Highlighting}
\end{Shaded}

\textbf{\emph{Question 4:}} How would you modify the \emph{sample()}
function to sample only three values at random, allowing a given number
or character to appear more than once?\\
\textbf{\emph{A/}}

You can also use R to plot functions just as most scientific calculators
do. But before coding plots, we will learn a few more R commands to
extract information from data frames or similar objects.

\hypertarget{step-7-practice-using-data-frames}{%
\subsubsection{Step 7: Practice using data
frames}\label{step-7-practice-using-data-frames}}

Very often we want to summarize, visualize, simulate or analyze existing
data, for that, one can start by using the data frame format in R. Here
we are going to explore functions using very basic data frames, to learn
more about how the data look like.

Note that in R studio, going to \textbf{Panel 2} and simply clicking on
a given data frame, will retrieve all its contents in a new tab in
\textbf{Panel 1}.

\textbf{\emph{Question 5:}} How would you modify one of this functions
(e.g. \emph{mean()}) to deal with missing values?\\
\textbf{\emph{A/}}

\hypertarget{step-8-practice-exporting-data-frame}{%
\subsubsection{Step 8: Practice exporting data
frame}\label{step-8-practice-exporting-data-frame}}

If you want to save/export the data frame you created in R, you just
need to call a function named \emph{write.csv}. To export your data
frame, just name the file. I recommend you save your traditional data
files (aka. \emph{excel-like} tables) to use in R as a
\textbf{\emph{.csv}} (\textbf{c}omma-\textbf{s}eparated \textbf{v}alues)
file format.

\begin{Shaded}
\begin{Highlighting}[]
\CommentTok{#Exporting data}
\KeywordTok{write.csv}\NormalTok{(df1, }\StringTok{"df_2022jul2.csv"}\NormalTok{) }
\end{Highlighting}
\end{Shaded}

\textbf{\emph{Question 6:}} Where is the file saved? Use the getwd()
function to find out. How can you export text or a list that was created
using R?\\
\textbf{\emph{A/}}

\hypertarget{step-9-practice-importing-data-in-r}{%
\subsubsection{Step 9: Practice importing data in
R}\label{step-9-practice-importing-data-in-r}}

Very often we have data that we want to import into R, there are several
formats that are compatible with R. Data tables can be generated in R or
other software, for example: excel, statistical software or google
sheets (free). When data is not generated in R, data, can still easily
be imported but there are distinct ways to import files depending the
format of the data to be imported. Common formats include data as: text
or \emph{.txt}, comma separated values or \textbf{\emph{.csv}} and/or as
excel \emph{.xlsx} (only the latter requires installing a package called
\emph{.openxlsx}. I recommend that you always save or import the data as
\textbf{\emph{.csv}}.

We previouly saved our data as a \emph{.csv}, to import these data, we
can just use the \textbf{read.csv()} function \textbf{\emph{-as long as
the file is contained in our current folder used as working directory.}}

\begin{Shaded}
\begin{Highlighting}[]
\CommentTok{#Import data as csv}
\NormalTok{imported_df1<-}\KeywordTok{read.csv}\NormalTok{(}\StringTok{"df_2022jul2.csv"}\NormalTok{, }\DataTypeTok{header=}\NormalTok{T)}
\end{Highlighting}
\end{Shaded}

\hypertarget{step-10-create-plots}{%
\subsubsection{Step 10: Create plots}\label{step-10-create-plots}}

We can use base R to create quick plots and identify patterns. We then
can export these as a \textbf{\emph{.pdf}} file format. Plotting is one
of the biggest strengths of R. The possibilities are unlimited and there
is a universe of tutorials out there to show you how to customize plots.
Over the years I have moved on from base R into using the package
\emph{ggplot2} which we will explore next.

\begin{Shaded}
\begin{Highlighting}[]
\CommentTok{#First: Plotting with base R}
\CommentTok{#Plot 1 continuous variable}
\KeywordTok{hist}\NormalTok{(}\KeywordTok{sample}\NormalTok{(}\KeywordTok{seq}\NormalTok{(}\DecValTok{1}\OperatorTok{:}\DecValTok{100}\NormalTok{),}\DecValTok{200}\NormalTok{, }\DataTypeTok{replace=}\OtherTok{TRUE}\NormalTok{), }\DataTypeTok{breaks=}\DecValTok{25}\NormalTok{, }\DataTypeTok{main=}\StringTok{""}\NormalTok{) }\CommentTok{#Note Question 7}

\CommentTok{#Plot 2 continuous variables}
\KeywordTok{plot}\NormalTok{(df1}\OperatorTok{$}\NormalTok{Cont1}\OperatorTok{~}\NormalTok{df1}\OperatorTok{$}\NormalTok{Cont2)}

\CommentTok{#Plot 1 continuous variable and 1 Binary/Categorical}
\KeywordTok{boxplot}\NormalTok{(df1}\OperatorTok{$}\NormalTok{Cont1}\OperatorTok{~}\NormalTok{df1}\OperatorTok{$}\NormalTok{Binary1)}

\CommentTok{#To export a pdf file with multiple figures:}
\KeywordTok{pdf}\NormalTok{(}\StringTok{"plotsA_2022jul2.pdf"}\NormalTok{)}
\KeywordTok{hist}\NormalTok{(}\KeywordTok{sample}\NormalTok{(}\KeywordTok{seq}\NormalTok{(}\DecValTok{1}\OperatorTok{:}\DecValTok{100}\NormalTok{),}\DecValTok{200}\NormalTok{, }\DataTypeTok{replace=}\OtherTok{TRUE}\NormalTok{), }\DataTypeTok{breaks=}\DecValTok{25}\NormalTok{, }\DataTypeTok{main=}\StringTok{""}\NormalTok{)}
\KeywordTok{plot}\NormalTok{(df1}\OperatorTok{$}\NormalTok{Cont1}\OperatorTok{~}\NormalTok{df1}\OperatorTok{$}\NormalTok{Cont2)}
\KeywordTok{boxplot}\NormalTok{(df1}\OperatorTok{$}\NormalTok{Cont1}\OperatorTok{~}\NormalTok{df1}\OperatorTok{$}\NormalTok{Binary1)}
\KeywordTok{dev.off}\NormalTok{()}
\end{Highlighting}
\end{Shaded}

\textbf{\emph{Question 7:}} What happens when we remove \emph{main=""}.
How would you go about to rename your y axis label?\\
\textbf{\emph{A/}}

Next we will use data on the survival status of individual passengers on
board of the \textbf{\emph{Titanic}} ship. Visit my google drive to
download the titanic data set \textbf{.csv} format:
\emph{\url{https://drive.google.com/file/d/1jhGvXkYSrtsgmEnZ2GBR6AzBy1BUGbNg/view?usp=sharing}}

After successfully downloading \emph{trains.csv} file you would be able
to use the \textbf{\emph{Upload}} option in \textbf{Panel 4}. Simply
choose the file to upload, and hit \textbf{\emph{Ok}}. Once you do this,
it will be available in your environment.

\textbf{\emph{Question 8:}} Did you get an error trying to upload this
data file? How did you go about to fix this error?\\
\textbf{\emph{A/}}

\begin{Shaded}
\begin{Highlighting}[]
\CommentTok{#Import the data set}
\NormalTok{titanic <-}\StringTok{ }\KeywordTok{read.csv}\NormalTok{(}\StringTok{'trains.csv'}\NormalTok{,}\DataTypeTok{stringsAsFactors =}\NormalTok{ F, }\DataTypeTok{header =}\NormalTok{ T)}

\CommentTok{#Explore the data set using the functions we learned before}
\KeywordTok{head}\NormalTok{(titanic) }\CommentTok{#to see first 10 rows and column names}
\KeywordTok{summary}\NormalTok{(titanic) }\CommentTok{#to get a summary of each column}
\CommentTok{#***Note that R describes missing data as "NA"}

\CommentTok{#Plot to summarize passenger age with base R}
\KeywordTok{hist}\NormalTok{(}\KeywordTok{na.omit}\NormalTok{(titanic}\OperatorTok{$}\NormalTok{age),}\DataTypeTok{col=}\KeywordTok{rainbow}\NormalTok{(}\KeywordTok{length}\NormalTok{(}\DecValTok{1}\OperatorTok{:}\DecValTok{8}\NormalTok{)),}\DataTypeTok{main=}\StringTok{""}\NormalTok{,}\DataTypeTok{xlab=}\StringTok{"age"}\NormalTok{, }\DataTypeTok{ylab=}\StringTok{"count"}\NormalTok{)}
\end{Highlighting}
\end{Shaded}

\hypertarget{step-11-practice-using-ggplot2}{%
\subsubsection{Step 11: Practice using
ggplot2}\label{step-11-practice-using-ggplot2}}

Base R plots can be modified to look as pretty/detailed as you would
like them to look. However, packages like \textbf{ggplot2} make all the
heavy-lifting for you. Next we will learn how to make a few simple, yet
publication-level, \emph{nice looking plots} using \textbf{ggplot2} and
the data management package \textbf{dplyr}. This tutorial does not cover
data wrangling using dplyr, but there are many out there that do!

We just created a simple histogram using base R to use as comparison
with the same plot created using \emph{ggplot2} and associated packages.

\begin{Shaded}
\begin{Highlighting}[]
\CommentTok{#Now let's use dplyr and ggplot2}
\CommentTok{#Part 1}
\CommentTok{#Note that the code gets increasingly complex, as you try to make}
\CommentTok{#better-looking, customized figures in R}
\NormalTok{titanic }\OperatorTok\StringTok{ }\KeywordTok{head}\NormalTok{(}\DecValTok{10}\NormalTok{) }\CommentTok{#Overview of the first 10 rows.}
\NormalTok{titanic }\OperatorTok\StringTok{ }\KeywordTok{describe_tbl}\NormalTok{() }\CommentTok{#Describes data.}

\CommentTok{#Plot to summarize passenger age with ggplot}
\NormalTok{p1<-titanic }\OperatorTok\StringTok{ }\KeywordTok{drop_na}\NormalTok{() }\OperatorTok\StringTok{ }\KeywordTok{ggplot}\NormalTok{(}\KeywordTok{aes}\NormalTok{(}\DataTypeTok{x=}\NormalTok{age))}\OperatorTok{+}
\StringTok{  }\KeywordTok{geom_histogram}\NormalTok{(}\DataTypeTok{fill=}\KeywordTok{rainbow}\NormalTok{(}\KeywordTok{length}\NormalTok{(}\DecValTok{1}\OperatorTok{:}\DecValTok{8}\NormalTok{)),}\DataTypeTok{bins=}\DecValTok{8}\NormalTok{,}\DataTypeTok{show.legend =} \OtherTok{FALSE}\NormalTok{)}\OperatorTok{+}
\StringTok{  }\KeywordTok{labs}\NormalTok{(}\DataTypeTok{x =} \StringTok{"age"}\NormalTok{)}
\NormalTok{p1}
\end{Highlighting}
\end{Shaded}

The histogram created using this simple ggplot Rcode is more attractive,
but we can do a lot more to customize this plot. For instance you can
try adding different themes (e.g., classic, bw, minimal). Simply type
\textbf{\emph{theme}} followed by tab on the console to learn about the
different themes and how to add them to an existing ggplot.

\hypertarget{step-12-create-publication-ready-plots-using-ggplot2-and-dplyr}{%
\subsubsection{Step 12: Create publication-ready plots using ggplot2 and
dplyr}\label{step-12-create-publication-ready-plots-using-ggplot2-and-dplyr}}

\begin{Shaded}
\begin{Highlighting}[]
\CommentTok{#Part 2}
\CommentTok{#We can combine multiple categorical and continuous variables}
\NormalTok{p2<-}\KeywordTok{ggplot}\NormalTok{(}\DataTypeTok{data=}\KeywordTok{na.omit}\NormalTok{(titanic),}
           \KeywordTok{aes}\NormalTok{(}\DataTypeTok{x=}\KeywordTok{factor}\NormalTok{(sex),}\DataTypeTok{y=}\NormalTok{age,}\DataTypeTok{fill=}\KeywordTok{factor}\NormalTok{(survived)))}\OperatorTok{+}
\StringTok{  }\KeywordTok{geom_boxplot}\NormalTok{()}\OperatorTok{+}
\StringTok{  }\KeywordTok{labs}\NormalTok{(}\DataTypeTok{title=}\StringTok{"Survival of passengers by sex, age and class"}\NormalTok{,}
       \DataTypeTok{subtitle=}\StringTok{"Not everyone had equal chances of surviving"}\NormalTok{, }
       \DataTypeTok{caption=}\StringTok{"Plot by J.Golcher"}\NormalTok{) }\OperatorTok{+}
\StringTok{  }\KeywordTok{xlab}\NormalTok{(}\StringTok{"Passenger class"}\NormalTok{)}\OperatorTok{+}
\StringTok{  }\KeywordTok{ylab}\NormalTok{(}\StringTok{"age"}\NormalTok{)}\OperatorTok{+}
\StringTok{  }\KeywordTok{facet_wrap}\NormalTok{(}\OperatorTok{~}\KeywordTok{factor}\NormalTok{(pclass))}\OperatorTok{+}
\StringTok{  }\KeywordTok{scale_fill_manual}\NormalTok{(}\DataTypeTok{values=}\KeywordTok{c}\NormalTok{(}\StringTok{"mediumvioletred"}\NormalTok{,}\StringTok{"cornflowerblue"}\NormalTok{),}
                    \DataTypeTok{name=}\StringTok{""}\NormalTok{,}\DataTypeTok{breaks=}\KeywordTok{c}\NormalTok{(}\StringTok{"0"}\NormalTok{, }\StringTok{"1"}\NormalTok{),}\DataTypeTok{labels=}\KeywordTok{c}\NormalTok{(}\StringTok{"died"}\NormalTok{, }\StringTok{"survived"}\NormalTok{))}\OperatorTok{+}
\StringTok{  }\KeywordTok{theme}\NormalTok{(}\DataTypeTok{plot.title =} \KeywordTok{element_text}\NormalTok{(}\DataTypeTok{face =} \StringTok{"bold"}\NormalTok{))}

\NormalTok{p2}
\end{Highlighting}
\end{Shaded}

Using ggplot, we can quickly make clear figures that help data
interpretation. Multiple data visualization R tutorials and cheat sheets
exist in the internet. If you are interested in how to go from beginner
to pro, again \textbf{\emph{practice makes perfect}}. There are also new
packages that make it easier to explore patterns from data, one of them
is the package \textbf{\emph{explore}}. This part of the tutorial is
adapted from a 2022 tutorial written by Roland Crasser that can be
accessed here:
\emph{\url{https://cran.r-project.org/web/packages/explore/vignettes/explore_titanic.html}}

\begin{Shaded}
\begin{Highlighting}[]
\CommentTok{#Part 3}
\CommentTok{#Let's explore functions and plots in explore}
\CommentTok{#}
\CommentTok{#First let's make age a categorical variable:}
\NormalTok{titanic}\OperatorTok{$}\NormalTok{age_cat <-}\StringTok{ }\KeywordTok{cut}\NormalTok{(titanic}\OperatorTok{$}\NormalTok{age,}\DataTypeTok{breaks=}\KeywordTok{c}\NormalTok{(}\DecValTok{0}\NormalTok{,}\DecValTok{19}\NormalTok{,}\DecValTok{60}\NormalTok{,}\DecValTok{90}\NormalTok{),}
                       \DataTypeTok{labels=}\KeywordTok{c}\NormalTok{(}\StringTok{"children"}\NormalTok{,}\StringTok{"adult"}\NormalTok{,}\StringTok{"senior"}\NormalTok{))}

\CommentTok{#This is perhaps unnecessary but just to be sure}
\NormalTok{titanic2 <-}\StringTok{ }\KeywordTok{tibble}\NormalTok{(titanic) }\CommentTok{#data frame into tibble}

\CommentTok{#We would like to reorganize our data as counts per categories}
\NormalTok{titan_sum<-titanic2 }\OperatorTok\StringTok{ }\KeywordTok{group_by}\NormalTok{(pclass, survived,age_cat,sex) }\OperatorTok\StringTok{ }\NormalTok{dplyr}\OperatorTok{::}\KeywordTok{count}\NormalTok{()}

\NormalTok{titan_sum }\OperatorTok\StringTok{ }\KeywordTok{describe_tbl}\NormalTok{(}\DataTypeTok{n=}\NormalTok{n) }\CommentTok{#describes data and NA}

\CommentTok{#Creating plots using explore}
\NormalTok{p3<-titan_sum }\OperatorTok\StringTok{ }\KeywordTok{explore}\NormalTok{(age_cat, }\DataTypeTok{n =}\NormalTok{ n) }\CommentTok{#passenger class distrib.}
\NormalTok{p4<-titan_sum }\OperatorTok\StringTok{ }\KeywordTok{explore}\NormalTok{(age_cat, }\DataTypeTok{target =}\NormalTok{ survived, }\DataTypeTok{n =}\NormalTok{ n)}
\NormalTok{p5<-titan_sum }\OperatorTok\StringTok{ }\KeywordTok{explore_all}\NormalTok{(}\DataTypeTok{n =}\NormalTok{ n) }\CommentTok{#To see counts for all vars.}
\NormalTok{p6<-titan_sum }\OperatorTok\StringTok{ }\KeywordTok{explore_all}\NormalTok{(}\DataTypeTok{target =}\NormalTok{ survived, }\DataTypeTok{n =}\NormalTok{ n, }\DataTypeTok{split =} \OtherTok{TRUE}\NormalTok{)}

\NormalTok{p3}
\NormalTok{p4}
\NormalTok{p5}
\NormalTok{p6}
\end{Highlighting}
\end{Shaded}

\hypertarget{step-13-export-ggplot2-charts}{%
\subsubsection{Step 13: Export ggplot2
charts}\label{step-13-export-ggplot2-charts}}

In this example we went beyond simply plotting using base R to practice
data re-organization with advanced visualization. Lastly we will learn
how to arrange and export ggplot figures using the package
\emph{patchwork}.

\begin{Shaded}
\begin{Highlighting}[]
\CommentTok{#Part 4}
\CommentTok{#A package to save ggplots}
\KeywordTok{library}\NormalTok{(patchwork)}
\NormalTok{p<-(p1)}\OperatorTok{/}\NormalTok{(p2)}
\CommentTok{#Save plots}
\KeywordTok{ggsave}\NormalTok{(}\StringTok{"plotsB_2022jul2.pdf"}\NormalTok{, p)}
\end{Highlighting}
\end{Shaded}

\textbf{\emph{Question 9:}} How would you modify this code to save the
plots next to one another using \emph{patchwork} (instead of one on top
ot the other).\\
\textbf{\emph{A/}}

Now we have explored how to make a few plots using base R and some of
the newest plotting options available associated to the package
\emph{ggplot2}. Still, remember that this is the very tiny tip of the
iceberg. It will be up to you to take it from here and make significant
progress in your R journey coding and making plots. There are almost too
many resources online to learn how to visualize data using R, but as a
start visit: \emph{\url{https://www.r-graph-gallery.com/}} to get
inspired with a gallery of different types of charts.

\hypertarget{step-14-run-a-simple-statistical-test}{%
\subsubsection{Step 14: Run a simple statistical
test}\label{step-14-run-a-simple-statistical-test}}

As a biologist, one of the first statistical tests we learn is how to
calculate a Chi-squared value. For instance, such tests are used in the
context of transmission genetics, together with the Chi-squared value
distribution table to test for significance rejecting a null hypothesis
of observations deviating or not from expected. As you already guessed
it, R can do this for you and much more. We will start by learning how
to do this test using the console, and review how to interpret results
of the \emph{chisq.test()} function in base R using the Titanic toy data
set.

\textbf{\emph{Question 10:}} What would be \textbf{1)}the null and
\textbf{2)}the alternative hypotheses for a Chi-squared test looking at
passenger survival by class? After running the code below:
\textbf{3)}Was the p-value significant? \textbf{4)} What does this
mean?\\
\textbf{\emph{A/}}

\begin{Shaded}
\begin{Highlighting}[]
\CommentTok{#Chi-squared test using R}
\NormalTok{res1<-}\KeywordTok{chisq.test}\NormalTok{(titanic}\OperatorTok{$}\NormalTok{pclass, titanic}\OperatorTok{$}\NormalTok{survived)}
\NormalTok{res1}
\NormalTok{res1}\OperatorTok{$}\NormalTok{observed}
\NormalTok{res1}\OperatorTok{$}\NormalTok{expected}
\end{Highlighting}
\end{Shaded}

\hypertarget{step-15-cite-r-and-rpackages}{%
\subsubsection{Step 15: Cite R and
Rpackages}\label{step-15-cite-r-and-rpackages}}

It is key to give credit to the generous people who develop packages for
free, often with public funds from universities and other research
institutions. Providing authors with the credit they deserve also
encourages funding agencies to support their work, which ultimately
benefits the entire community of R users.

\begin{Shaded}
\begin{Highlighting}[]
\KeywordTok{citation}\NormalTok{() }\CommentTok{#To cite R}
\KeywordTok{R.Version}\NormalTok{() }\CommentTok{#Figure out R version}
\KeywordTok{RStudio.Version}\NormalTok{() }\CommentTok{#To cite R studio}
\CommentTok{#To cite all the packages  used in this tutorial:}
\KeywordTok{citation}\NormalTok{(}\StringTok{"ggplot2"}\NormalTok{)}
\KeywordTok{citation}\NormalTok{(}\StringTok{"dplyr"}\NormalTok{)}
\KeywordTok{citation}\NormalTok{(}\StringTok{"tibble"}\NormalTok{)}
\KeywordTok{citation}\NormalTok{(}\StringTok{"explore"}\NormalTok{)}
\KeywordTok{citation}\NormalTok{(}\StringTok{"patchwork"}\NormalTok{)}
\end{Highlighting}
\end{Shaded}

\emph{This mini-workshop was modified from the 2018 Workshop notes Intro
to R for fisheries researchers in Tanzania co-taught by Dr.~Jessica Rick
and Dr.~Jimena Golcher-Benavides. I would like to thank the generous R
\& R learning community for all the high-quality learning resources
freely available online.}

For more learning resources on how to use R \& R Studio, check out these
guidelines written by Thomas Mock:
\emph{\url{https://www.rstudio.com/resources/webinars/a-gentle-introduction-to-tidy-statistics-in-r/}}.

\end{document}
